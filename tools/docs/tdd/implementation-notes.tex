\documentclass[a4paper,twoside]{article}

%setup the page layout
\usepackage{fancyhdr}
\pagestyle{fancy}
\fancyhf{}
\fancyhead[CE]{M3X}
\fancyhead[CO]{\today}
\fancyfoot[LE,RO]{\thepage}

%setup the title page
\title{M3X \\ Implementation notes}
\author{Jyrki Saarinen\\
   \texttt{jyrki.saarinen@ardites.com}
   }
\date{\today}

%start the document
\begin{document}
\maketitle

\section{Introduction}

This document is meant mainly for the M3X project developers. 
Currently (\date{\today}) it is briefly discussed how things 
have been implemented, and what is missing.

\section{The m3x.m3g package}

This package follows the class hierarchy specified in the M3G 1.0 specification.

\texttt{M3GObject}, \texttt{Section} classes and the abstract \texttt{Object3D} 
class are the most important pieces. All concrete M3G classes are derived
from the \texttt{Object3D}, as it is modeled in the specification.

Between the \texttt{Section} and \texttt{Object3D} classes there is an
\texttt{ObjectChunk} class 'layer'. The class serves as a data container only.
The rationale between this design decision was that having the responsibilites
of \texttt{ObjectChunk} in \texttt{Object3D} would have meant major amount
of bookkeeping in \texttt{Object3D} class. Now the bookkeeping is done by the
\texttt{java.io} streams instead of manual bookkeeping.
Also this decision follows the M3G specification nicely.

All M3G classes implement the interface \texttt{M3GSerializable}. This interface
specifies serialization and deserialization to and from streams. Concrete classes
implement \texttt{M3GTypedObject} which is derived from \texttt{M3GSerializable}.

\subsection{What is implemented}

All M3G 1.0 classes are implemented following the specification. There are
also JUnit test cases for every M3G class, and an end-to-end test (from M3G object
to serialized form and back).

\subsection{What is to be done}

Everything should be done by now (\date{\today}).

\section{The m3x.translation package}

\subsection{What is implemented}

Translation from M3X to M3G is implemented for all classes.

\subsection{What is to be done}

All M3X to M3G translation classes do not have a JUnit test cases.
All M3G to M3X translation methods are missing from the translator classes
(returning \texttt{null}).

\end{document}
